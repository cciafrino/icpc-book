\chapter{Graph}

\section{Fundamentals}
	\kactlimport{BellmanFord.h}
	\kactlimport{FloydWarshall.h}
	\kactlimport{TopoSort.h}
    \kactlimport{CutVertices.h}
    \kactlimport{Bridges.h}
    \kactlimport{Dijkstra.cpp}
    \kactlimport{Prim.h}
    \kactlimport{Kruskal.h}
    \kactlimport{SPFA.h}
	\kactlimport{EulerWalk.h}

\section{Network flow}
	\kactlimport{PushRelabel.h}
	\kactlimport{Dinitz.h}
	\kactlimport{HLPP.h}
	\kactlimport{EdmondsKarp.h}
	\kactlimport{MinCut.h}
	\kactlimport{MinCostMaxFlow.h}
	\kactlimport{StoerWagner.h}
	\kactlimport{GomoryHu.h}
  
\section{Matching}
	\kactlimport{HopcroftKarp.h}
	\kactlimport{MaxBipartiteMatching.h}
    \kactlimport{WeightedMatching.h}
	\kactlimport{GeneralMatching.h}
	\kactlimport{MaximumIndependentSet.h}
	\kactlimport{MinimumVertexCover.h}
	\kactlimport{MinimumEdgeCover.h}
	\kactlimport{MinimumPathCover.h}
	%\kactlimport{Koenig.cpp}

\section{DFS algorithms}
    \kactlimport{CentroidDecomposition.cpp}
	\kactlimport{Tarjan.h}
	%\kactlimport{Kosaraju.h}
	\kactlimport{BiconnectedComponents.h}
	\kactlimport{2sat.h}
	\kactlimport{Cycles.h}

\section{Heuristics}
	\kactlimport{MaximalCliques.h}
	\kactlimport{MaximumClique.h}
	\kactlimport{Cycle-Counting.cpp}

\section{Trees}
    \kactlimport{Tree.h}
	\kactlimport{LCA.cpp}
	\kactlimport{LCA.h}
	\kactlimport{CompressTree.h}
    %\kactlimport{HLD.h}
    \kactlimport{Heavylight.h}
    \kactlimport{Tree-Isomorphism.h}
    \kactlimport{LineTree.h}
	%\kactlimport{LinkCutTree.h}
	%\kactlimport{MatrixTree.h}
    \subsection{Sqrt Decomposition}
		HLD generally suffices. If not, here are some common strategies:
		\begin{itemize}
			\item Rebuild the tree after every $\sqrt N$ queries. 
			\item Consider vertices with $>$ or $<\sqrt N$ degree separately. 
			\item For subtree updates, note that there are $O(\sqrt N)$ distinct sizes among child subtrees of any vertex.
		\end{itemize}

		\textbf{Block Tree:} Use a DFS to split edges into contiguous groups of size $\sqrt N$ to $2\sqrt N.$

		\textbf{Mo's Algorithm for Tree Paths:} Maintain an array of vertices where each one appears twice, once when a DFS enters the vertex (\texttt{st}) and one when the DFS exists (\texttt{en}). For a tree path $u\leftrightarrow v$ such that \texttt{st[u]<st[v]},

		\begin{itemize}
		\item If $u$ is an ancestor of $v,$ query \texttt{[st[u],st[v]]}.
		\item Otherwise, query $\texttt{[en[u],st[v]]}$ and consider $lca(u,v)$ separately.
		\end{itemize}
    
\section{Functional Graphs}
    \kactlimport{Lumberjack.h}
    \kactlimport{Lumberjack2.h}
\section{Other}
    \kactlimport{kthShortestPath.h}
    \kactlimport{MatrixTreeMST.h}
    \kactlimport{ManhattanMST.h}
    %\kactlimport{SteinerTree.h}
    \kactlimport{Pruefer.cpp}    
    \kactlimport{ErdosGallai.h}
    \kactlimport{MisraGries.h}
	\kactlimport{Directed-MST.cpp}
	\kactlimport{Graph-Negative-Cycle.cpp}
    %\kactlimport{TransitiveClosure.h}

\section{Theorems}
\subsection{Vizing's Thereom}
Dado um grafo~$G$, seja~$\delta$ o maior grau de um vértice. Então~$G$ tem número cromático de aresta~$\delta$ ou~$\delta + 1$.

\subsection{Euler's theorem}
Sendo~$V$,~$A$ e~$F$ as quantidades de vértices, arestas e faces de um grafo planar conexo,
$V - A + F = 2$.

\subsection{Hall's Marriage theorem}
Dado um grafo bipartido com classes~$V_1$ e~$V_2$, para~$S$~$\subset$~$V_1$ seja~$N(S)$ o conjunto de todos os vértices vizinhos a algum elemento de~$S$. Um emparelhamento de~$V_1$ em~$V_2$ é um conjunto de arestas disjuntas cujas extremidades estão em classes diferentes. Então existe um emparelhamento completo de~$V_1$ em~$V_2$ sse $|N(S)| \geq |S|$ \forall $ S \subset V_1 $.
    
\iffalse % commenting this block
\subsection{Maximum Density Subgraph}
Given (weighted) undirected graph $G$. Binary search density. If $g$ is
current density, construct flow network: $(S, u, m)$, $(u, T,
m+2g-d_u)$, $(u,v,1)$, where $m$ is a large constant (larger than sum
of edge weights). Run floating-point max-flow. If minimum cut has empty
$S$-component, then maximum density is smaller than $g$, otherwise it's
larger. Distance between valid densities is at least $1/(n(n-1))$. Edge
case when density is $0$. This also works for weighted graphs by
replacing $d_u$ by the weighted degree, and doing more iterations (if
weights are not integers).
\fi

\subsection{Dilworth's theorem}
Em todo conjunto parcialmente ordenado, a quantidade máxima de elementos de uma anticadeia é igual à  quatidade mínima de cadeias disjuntas que cobrem o conjunto.

\subsection{Conjunto Independente de Peso Máximo num Grafo Bipartido}
É o mesmo que a cobertura de peso mínimo. Podemos resolver criando uma rede de fluxo com arestas $(S,u,w(u))$ para $u\in L$,
$(v,T,w(v))$ para $v\in R$ e $(u,v,\infty)$ para $(u,v)\in E$. O corte mínimo de $S$ a $T$ é a resposta. Vértices adjacentes a uma aresta de corte estão na cobertura de vértices.

\iffalse % commenting this block
\subsection{Fecho de Peso Máximo}
Dado um digrafo $G$ com peso nos vértices. Transforme $G$ numa rede de fluxo, colocando o peso de cada aresta como $\infty$. Adicione vértices $S,T$. Para cada vértice $v$ de peso $w$, adicione uma aresta $(S, v, w)$ se $w\geq 0$, ou a aresta $(v,T,-w)$ se $w<0$. A soma de todos os pesos positivos menos o corte mínimo $c(S,T)$ é a resposta. Vértices que são alcançados a partir de $S$ estão no fecho. O fecho de peso máximo é o mesmo que o complemento do fecho de peso mínimo num grafo com as arestas invertidas.
\fi


\subsection{Tutte-Berge formula}
The theorem states that the size of a maximum matching of a graph $G=(V,E)$ equals ${\frac {1}{2}}\min _{U\subseteq V}\left(|U|- \texttt{odd}(G-U)+|V|\right)\,,$ where \texttt{odd}(H) counts how many of the connected components of the graph $H$ have an odd number of vertices.

\subsection{Tutte's theorem}
Um grafo $G = (V, A)$ tem um emparelhamento perfeito sse para todo subconjunto $U$ de $V$, o subgrafo induzido por $V \setminus U$ tem no máximo $|U|$ componentes conexas com uma quantidade ímpar de vértices.

\subsection{Menger's theorem}
\item Para vértices:
Um grafo é k-conexo sse todo par de vértices é conectado por pelo menos k caminhos sem vértices intermediários em comum.
\item Para arestas:
Um grafo é dito~$k$-aresta-conexo se a retirada de menos de~$k$ arestas do grafo o mantém conexo. Então um grafo é~$k$ -aresta-conexo sse para todo par de vértices~$u$ e~$v$, existem~$k$ caminhos que ligam~$u$ a~$v$ sem arestas em comum.

\subsection{Landau}
Existe um torneio com graus de saída~$d_1 \leq d_2 \leq \ldots \leq d_n$ sse:
\begin{itemize}
\item $d_1 + d_2 + \ldots + d_n = {n \choose 2}$\newline
\item $d_1 + d_2 + \ldots + d_k \geq {k \choose 2} \quad \forall 1 \leq k \leq n.$
\end{itemize}
Para construir, fazemos~1 apontar para~$2, 3, \ldots, d_1 + 1$ e seguimos recursivamente.

\subsection{Erdős–Gallai theorem}
		% Source: https://en.wikipedia.org/wiki/Erd%C5%91s%E2%80%93Gallai_theorem
		% Test: stress-tests/graph/matrix-tree.cpp
		A simple graph with node degrees $d_1 \ge \dots \ge d_n$ exists iff $d_1 + \dots + d_n$ is even and for every $k = 1\dots n$,
		\[ \sum _{i=1}^{k}d_{i}\leq k(k-1)+\sum _{i=k+1}^{n}\min(d_{i},k). \]

\iffalse
\subsection{Synchronizing word problem}
A DFA has a synchronizing word (an input sequence that moves all states
to the same state) iff.\ each pair of states has a synchronizing word.
That can be checked using reverse DFS over pairs of states. Finding the
shortest synchronizing word is NP-complete.
\fi

\subsection{Turán's theorem} 
Nenhum grafo com $n$ vértices que é $K_{r+1}$-livre pode ter mais arestas do que o grafo de $Tur\acute{a}n$: Um grafo completo $k$-partido com conjuntos de tamanho mais próximo possível.


\subsection{Dirac's theorem} Seja $G$ um grafo com $n$ vértices, cada um com grau pelo menos $n/2$. Então $G$ é hamiltoniano.

\subsection{Ore's theorem} Seja $G$ um grafo simples de ordem $n \geq 3$ tal que 
\[g(u) + g(v) \geq n \] para todo par $u$,$v$ de vértices não adjacentes, então $G$ é hamiltoniano.

\subsection{Eulerian Cycles}
A quantidade de ciclos $Eulerianos$ num digrafo $G$ é:\newline
		\begin{equation*}
			t_w(G) \prod_{v \in G} (\deg v - 1)!,
		\end{equation*}\newline
		onde $t_w(G)$ é a quantidade de arborescências (árvore geradora direcionada) enraizada em $w$: $t_w(G) = \det\left( q_{ij} \right)_{i,j \neq w}$, with $q_{ij} = [i=j] \mathrm{indeg}(i) - \# \set{(i,j) \in E}$.

\subsection{Matroid Intersection theorem}
Sejam~$M_1 = (E, I_1)$ e~$M_2 = (E, I_2)$ matróides. Então
$ \max\limits_{S \in I_1 \cap I_2}{|S|} = \min\limits_{U \subseteq E}{r_1(U) + r_2(E \setminus U)}.$

\subsection{Number of Spanning Trees}
		% I.e. matrix-tree theorem.
		% Source: https://en.wikipedia.org/wiki/Kirchhoff%27s_theorem
		% Test: stress-tests/graph/matrix-tree.cpp
		Create an $N\times N$ matrix \texttt{mat}, and for each edge $a \rightarrow b \in G$, do
		\texttt{mat[a][b]--, mat[b][b]++} (and \texttt{mat[b][a]--, mat[a][a]++} if $G$ is undirected).
		Remove the $i$th row and column and take the determinant; this yields the number of directed spanning trees rooted at $i$
		(if $G$ is undirected, remove any row/column).\newline

\subsection{König-Egervary theorem}
    Em todo grafo bipartido~$G$, a quantidade de arestas no emparelhamento máximo é maior ou igual à quantidade de vértices na cobertura mínima. Ou seja, para todo~$G$,~$\alpha(G) \geq \beta(G)$. Note que isso prova que~$\alpha(G) = \beta(G)$ para grafos bipartidos.

\


\subsection{Tutte Matrix}
	\begin{itemize}
	\item A graph has a perfect matching iff the $Tutte$ matrix has a non-zero determinant.

	\item The rank of the $Tutte$ matrix is equal to twice the size of the maximum matching. The maximum cost matching can be found by polynomial interpolation.
	\end{itemize}
