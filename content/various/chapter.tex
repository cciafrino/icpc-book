\chapter{Various}

\section{Intervals}
	\kactlimport{IntervalContainer.h}
	\kactlimport{IntervalCover.h}
	\kactlimport{ConstantIntervals.h}
    
\section{Misc. algorithms}
	\kactlimport{TernarySearch.h}
	\kactlimport{LowerBound.h}
	\kactlimport{UpperBound.h}	
	\kactlimport{MergeSort.h}
	\kactlimport{CoordCompression.h}
	\kactlimport{CountTriangles.h}	
	\kactlimport{sqrt.h}
	\kactlimport{Karatsuba.h}
	\kactlimport{CountInversions.h}
	\kactlimport{Histogram.h}
	\kactlimport{DateManipulation.h}	
	\kactlimport{MagicSquare.h}	
    \kactlimport{FindPattern.h}
    \kactlimport{NQueens.cpp}
    \kactlimport{SudokuSolver.cpp}    
    \kactlimport{FloydCycle.h}
    \kactlimport{SlidingWindow.h}
	%\newpage

\section{Dynamic programming}
	\kactlimport{DivideAndConquerDP.h}
	\kactlimport{KnuthDP.h}
	\kactlimport{ConvexHullTrick.h}
	\kactlimport{Coin.h}
	\kactlimport{MinCoin.h}		
	\kactlimport{EditDistance.h}
	\kactlimport{LIS.h}
	\kactlimport{LIS2.h}
	\kactlimport{LCS.h}
	\kactlimport{Knapsack.h}
	\kactlimport{01Knapsack.h}
	\kactlimport{LargeKnapsack.h}
	\kactlimport{KnapsackUnbounded.h}	
	\kactlimport{KnapsackBounded.h}
	\kactlimport{KnapsackBitset.h}	
	\kactlimport{TSP.h}
	\kactlimport{DistinctSubsequences.h}

\section{Debugging tricks}
	\begin{itemize}
		\item \texttt{signal(SIGSEGV, [](int) \{ \_Exit(0); \});} converts segfaults into Wrong Answers.
			Similarly one can catch SIGABRT (assertion failures) and SIGFPE (zero divisions).
			\texttt{\_GLIBCXX\_DEBUG} violations generate SIGABRT (or SIGSEGV on gcc 5.4.0 apparently).
		\item \texttt{feenableexcept(29);} kills the program on NaNs (\texttt 1), 0-divs (\texttt 4), infinities (\texttt 8) and denormals (\texttt{16}).
	\end{itemize}

\section{Optimization tricks}
	\subsection{Bit hacks}
		\begin{itemize}
			\item \texttt{x \& -x} is the least bit in \texttt{x}.
			\item \texttt{for (int x = m; x; ) \{ --x \&= m; ... \}} loops over all subset masks of \texttt{m} (except \texttt{m} itself).
			\item \texttt{c = x\&-x, r = x+c; (((r\^{}x) >> 2)/c) | r} is the next number after \texttt{x} with the same number of bits set.
			\item \texttt{ rep(b,0,K) rep(i,0,(1 << K)) if (i \& 1 << b) D[i] += D[i\^{}(1 << b)]; } computes all sums of subsets.
		\end{itemize}
	\subsection{Pragmas}
		\begin{itemize}
			\item \lstinline{#pragma GCC optimize ("Ofast")} will make GCC auto-vectorize for loops and optimizes floating points better (assumes associativity and turns off denormals).
			\item \lstinline{#pragma GCC target ("avx,avx2")} can double performance of vectorized code, but causes crashes on old machines.
			\item \lstinline{#pragma GCC optimize ("trapv")} kills the program on integer overflows (but is really slow).
		\end{itemize}
	\kactlimport{BumpAllocator.h}
	\kactlimport{SmallPtr.h}
	\kactlimport{BumpAllocatorSTL.h}
	\kactlimport{Unrolling.h}
	\kactlimport{SIMD.h}
	\kactlimport{Hashmap.h}
\section {Bit Twiddling Hack}
    \kactlimport{Hacks.h}
	\kactlimport{Bitset.h}
\section {Random Numbers}
    \kactlimport{RandomNumbers.h}
\section{Other languages}
	\kactlimport{Main.java}
	\kactlimport{MiscJava.java}
	\subsection{BigInteger}
	\paragraph{BigInteger}
To convert to a \texttt{BigInteger}, use \texttt{BigInteger.valueOf (int)} or \texttt{BigInteger (String, radix)}.

To convert from a \texttt{BigInteger}, use \texttt{.intValue ()}, \texttt{.longValue ()}, \texttt{.toString (radix)}.

Common unary operations include \texttt{.abs ()}, \texttt{.negate ()}, \texttt{.not ()}.

Common binary operations include \texttt{.max}, \texttt{.min}, \texttt{.add}, \texttt{.subtract}, \texttt{.multiply}, \texttt{.divide}, \texttt{.remainder}, \texttt{.gcd}, \texttt{.modInverse}, \texttt{.and}, \texttt{.or}, \texttt{.xor}, \texttt{.shiftLeft (int)}, \texttt{.shiftRight (int)}, \texttt{.pow (int)}, \texttt{.compareTo}.

Divide and remainder: \texttt{Biginteger[] .divideAndRemainder (Biginteger val)}.

Power module: \texttt{.modPow (BigInteger exponent, module)}.

Primality check: \texttt{.isProbablePrime (int certainty)}.\newline

