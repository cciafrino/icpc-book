\chapter{Numerical}
\kactlimport{polynomial.h}
\kactlimport{poly-roots.h}
\kactlimport{poly-interpolate.h}
\kactlimport{lagrange.h}
\kactlimport{berlekamp-massey.h}
\kactlimport{linear-recurrence.h}
\kactlimport{integrate.h}
\kactlimport{integrate-adaptive.h}
\kactlimport{gaussian-elimination.h}
\kactlimport{linear-solver-z2.h}
% \kactlimport{char-poly.h}
\kactlimport{simplex.h}
\kactlimport{tridiagonal.h}
% \kactlimport{polyominoes.h}
\section{Fourier transforms}
\kactlimport{fast-fourier-transform.h}
\kactlimport{fast-subset-transform.h}
% \kactlimport{finite-field-fft.h}	
% \kactlimport{poly-998244353.h}
\kactlimport{sum-of-powers.h}

\iffalse
\subsection{General linear recurrences}
  If $a_n = \sum_{k=0}^{n-1} a_k b_{n-k}$, then 
  $A(x) = \frac{a_0}{1-B(x)}$.

\subsection{Fast subset convolution}
  Given array $a_i$ of size $2^k$ calculate $b_i = \sum_{j \BitAnd i = i} a_j$.
\fi

\subsection{Duality}
  max $c^Tx$ sjt to $Ax \leq b$. Dual problem is min $b^Tx$ sjt to $A^Tx \geq c$. By strong duality, min max value coincides.

%  \subsection{Strong duality}
%  Given a linear problem $\Pi_{1}$: minimize $c^t x$, sjt to $Ax \leq b$, $x \geq 0$ we can define the linear problem dual standard $\Pi_{2}$ like the following: minimize $-b^t y$, sjt to $A^t y \geq c$. If $\Pi_{1}$ is satisfied then $\Pi_{2}$ is also satisfied and $c^t x = b^t y$. If $\Pi_{1}$ is not satisfied and unbounded, then $\Pi_{2}$ is not satisfied and unbounded. (OBS: Can't be both unbounded!)




%\subsection{Polyominoes} How many free (rotation, reflection), one-sided (rotation) and fixed $n$-ominoes are there?
%      \resizebox{\columnwidth}{!}{%
%       \begin{tabular}{|c|c|c|c|c|c|c|c|c|}
%        \hline
%        n&3&4&5&6&7&8&9&10 \\ \hline
%        free&2&5&12&35&108&369&1.285&4.655 \\ \hline
%        one-sided&2&7&18&60&196&704&2.500&9.189 \\ \hline
%        fixed&6&19&63&216&760&2.725&9.910&36.446 \\ \hline
%      \end{tabular}
%      }


\subsection{Generating functions}
  A list of generating functions for useful sequences:


  \begin{tabular}{|c|c|}
    \hline
    $(1,1,1,1,1,1,\ldots)$ & $\frac{1}{1-z}$ \\ \hline
    $(1,-1,1,-1,1,-1,\ldots)$ & $\frac{1}{1+z}$ \\ \hline
    $(1,0,1,0,1,0,\ldots)$ & $\frac{1}{1-z^2}$ \\ \hline
    $(1,0,\ldots,0,1,0,1,0,\ldots,0,1,0,\ldots)$ & $\frac{1}{1-z^2}$ \\ \hline
    $(1,2,3,4,5,6,\ldots)$ & $\frac{1}{(1-z)^2}$ \\ \hline
    $(1,\binom{m+1}{m},\binom{m+2}{m},\binom{m+3}{m},\ldots)$ & $\frac{1}{(1-z)^{m+1}}$ \\ \hline
    $(1,c,\binom{c+1}{2},\binom{c+2}{3},\ldots)$ & $\frac{1}{(1-z)^c}$ \\ \hline
    $(1,c,c^2, c^3, \ldots)$ & $\frac{1}{1-cz}$ \\ \hline
    $(0,1,\frac{1}{2},\frac{1}{3},\frac{1}{4},\ldots)$ & $\ln \frac{1}{1-z}$ \\ \hline
  \end{tabular}


A neat manipulation trick is:
\begin{equation*}
  \frac{1}{1-z}G(z) = \sum_{n}\sum_{k\leq n}g_kz^n
\end{equation*}


