\chapter{Numerical}
\kactlimport{GoldenSectionSearch.h}
\kactlimport{Polynomial.h}
\kactlimport{PolyRoots.h}
\kactlimport{PolyInterpolate.h}
\kactlimport{BerlekampMassey.h}
\kactlimport{LinearRecurrence.h}
\kactlimport{HillClimbing.h}
\kactlimport{Integrate.h}
\kactlimport{IntegrateAdaptive.h}
\kactlimport{Determinant.h}
\kactlimport{IntDeterminant.h}
\kactlimport{Simplex.h}
\kactlimport{Math-Simplex.cpp}
\kactlimport{SolveLinear.h}
\kactlimport{SolveLinear2.h}
\kactlimport{SolveLinearBinary.h}
\kactlimport{MatrixInverse.h}
\kactlimport{Tridiagonal.h}
\kactlimport{NewtonMethod.h}
\kactlimport{NewtonSQRT.h}
\kactlimport{MarkovChain.cpp}
\section{Fourier transforms}
	%\kactlimport{fft.cpp}
	%\kactlimport{poly-exp-log.cpp}
	\kactlimport{FastFourierTransform.h}
	\kactlimport{FastFourierTransformMod.h}	
	\kactlimport{NumberTheoreticTransform.h}
	\kactlimport{FastSubsetTransform.h}
\newpage

	\subsection{Generating functions}
	Ordinary (ogf): $A(x) := \sum_{n=0}^{\infty} a_i x^i$.

	Calculate product $c_n = \sum_{k=0}^{n} a_k b_{n-k}$ with FFT.

	Exponential (e.g.f.): $A(x) := \sum_{n=0}^{\infty} a_i x^i/i!$,

	$c_n = \sum_{k=0}^{n} \binom{n}{k} a_k b_{n-k} = n! \sum_{k=0}^{n} \frac{a_k}{k!} \frac{b_{n-k}}{(n-k)!}$ (use FFT).

\subsection{General linear recurrences}
	If $a_n = \sum_{k=0}^{n-1} a_k b_{n-k}$, then $A(x) = \frac{a_0}{1-B(x)}$.

\subsection{Inverse polynomial modulo $x^l$}
	Given $A(x)$, find $B(x)$ such that $A(x)B(x) = 1 + x^l Q(x)$ for some $Q(x)$.

	Step 1: Start with $B_0(x) = 1/a_0$

	Step 2: $B_{k+1}(x) = (-B_k(x)^2 A(x) + 2 B_k(x)) \mod x^{2^{k+1}}$.

\subsection{Fast subset convolution}
	Given array $a_i$ of size $2^k$ calculate $b_i = \sum_{j \BitAnd i = i} a_j$.
	

\subsection{Primitive Roots}
	It only exists when $n$ is $2, 4, p^k, 2p^k$, where $p$ odd prime.
	If $g$ is a primitive root, all primitive roots are of the form $g^k$
	where $k,\phi(p)$ are coprime (hence there are $\phi(\phi(p))$ primitive roots).

