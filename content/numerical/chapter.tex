\chapter{Numerical}
\kactlimport{GoldenSectionSearch.h}
\kactlimport{Polynomial.h}
\kactlimport{PolyRoots.h}
\kactlimport{PolyInterpolate.h}
\kactlimport{BerlekampMassey.h}
\kactlimport{LinearRecurrence.h}
\kactlimport{HillClimbing.h}
\kactlimport{Integrate.h}
\kactlimport{IntegrateAdaptive.h}
\kactlimport{Determinant.h}
\kactlimport{IntDeterminant.h}
\kactlimport{Elimination.h}
%\kactlimport{Simplex.h}
\kactlimport{Math-Simplex.cpp}
\kactlimport{SolveLinear.h}
\kactlimport{SolveLinear2.h}
\kactlimport{SolveLinearBinary.h}
\kactlimport{MatrixInverse.h}
\kactlimport{Tridiagonal.h}
\kactlimport{NewtonMethod.h}
\kactlimport{NewtonSQRT.h}
\kactlimport{MarkovChain.cpp}
\section{Fourier transforms}
%\kactlimport{fft.cpp}
%\kactlimport{poly-exp-log.cpp}
\kactlimport{FastFourierTransform.h}
\kactlimport{FastFourierTransformMod.h}	
\kactlimport{NumberTheoreticTransform.h}
\kactlimport{FastSubsetTransform.h}

\subsection{General linear recurrences}
  If $a_n = \sum_{k=0}^{n-1} a_k b_{n-k}$, then $A(x) = \frac{a_0}{1-B(x)}$.

\subsection{Inverse polynomial modulo $x^l$}
  Given $A(x)$, find $B(x)$ such that $A(x)B(x) = 1 + x^l Q(x)$ for some $Q(x)$.

  Step 1: Start with $B_0(x) = 1/a_0$

  Step 2: $B_{k+1}(x) = (-B_k(x)^2 A(x) + 2 B_k(x)) \mod x^{2^{k+1}}$.

\subsection{Fast subset convolution}
  Given array $a_i$ of size $2^k$ calculate $b_i = \sum_{j \BitAnd i = i} a_j$.
  
\subsection{Polyominoes} How many free (rotation, reflection), one-sided (rotation) and fixed $n$-ominoes are there?
      
      \resizebox{\columnwidth}{!}{%
       \begin{tabular}{|c|c|c|c|c|c|c|c|c|}
        \hline
        n&3&4&5&6&7&8&9&10 \\ \hline
        free&2&5&12&35&108&369&1.285&4.655 \\ \hline
        one-sided&2&7&18&60&196&704&2.500&9.189 \\ \hline
        fixed&6&19&63&216&760&2.725&9.910&36.446 \\ \hline
      \end{tabular}
      }


\subsection{Determinants}

              det(A) &= \sum_{\sigma \in S_n}\text{sgn}(\sigma)\prod_{i = 1}^n a_{i,\sigma(i)}
              %perm(A) &= \sum_{\sigma \in S_n} \prod_{i = 1}^n a_{i,\sigma(i)}\\
              %pf(A) &= \frac{1}{2^nn!}\sum_{\sigma \in S_{2n}} \text{sgn}(\sigma)\prod_{i = 1}^n a_{\sigma(2i-1),\sigma(2i)}\\ &= \sum_{M \in \text{PM}(n)} \text{sgn}(M) \prod_{(i,j) \in M} a_{i,j}

\subsection{Generating functions}
  A list of generating functions for useful sequences:

  \begin{tabular}{|c|c|}
    \hline
    $(1,1,1,1,1,1,\ldots)$ & $\frac{1}{1-z}$ \\ \hline
    $(1,-1,1,-1,1,-1,\ldots)$ & $\frac{1}{1+z}$ \\ \hline
    $(1,0,1,0,1,0,\ldots)$ & $\frac{1}{1-z^2}$ \\ \hline
    $(1,0,\ldots,0,1,0,1,0,\ldots,0,1,0,\ldots)$ & $\frac{1}{1-z^2}$ \\ \hline
    $(1,2,3,4,5,6,\ldots)$ & $\frac{1}{(1-z)^2}$ \\ \hline
    $(1,\binom{m+1}{m},\binom{m+2}{m},\binom{m+3}{m},\ldots)$ & $\frac{1}{(1-z)^{m+1}}$ \\ \hline
    $(1,c,\binom{c+1}{2},\binom{c+2}{3},\ldots)$ & $\frac{1}{(1-z)^c}$ \\ \hline
    $(1,c,c^2, c^3, \ldots)$ & $\frac{1}{1-cz}$ \\ \hline
    $(0,1,\frac{1}{2},\frac{1}{3},\frac{1}{4},\ldots)$ & $\ln \frac{1}{1-z}$ \\ \hline
  \end{tabular}

\


A neat manipulation trick is:
\begin{equation*}
  \frac{1}{1-z}G(z) = \sum_{n}\sum_{k\leq n}g_kz^n
\end{equation*}


\


\subsection{Table of non-trigonometric integrals}
      Some useful integrals are:
      
      \begin{tabular}{|c|c|}
        \hline
        $\int \frac{dx}{x^2 + a^2}$ & $\frac{1}{a} \arctan \frac{x}{a}$ \\ \hline
        $\int \frac{dx}{x^2 - a^2}$ & $\frac{1}{2a} \ln \frac{x - a}{x + a}$ \\ \hline
        $\int \frac{dx}{a^2 - x^2}$ & $\frac{1}{2a} \ln \frac{a + x}{a - x}$ \\ \hline
        $\int \frac{dx}{\sqrt{a^2 - x^2}}$ & $\arcsin \frac{x}{a}$ \\ \hline
        $\int \frac{dx}{\sqrt{x^2 - a^2}}$ & $\ln \left(u + \sqrt{x^2 - a^2}\right)$ \\ \hline
        $\int \frac{dx}{x \sqrt{x^2 - a^2}}$ & $\frac{1}{a} \text{arcsec} \left| \frac{u}{a} \right|$ \\ \hline
        $\int \frac{dx}{x \sqrt{x^2 + a^2}}$ & $-\frac{1}{a} \ln \left( \frac{a + \sqrt{x^2 + a^2}}{x} \right)$ \\ \hline
        $\int \frac{dx}{x \sqrt{a^2 + x^2}}$ & $-\frac{1}{a} \ln \left( \frac{a + \sqrt{a^2 - x^2}}{x} \right)$ \\ \hline
      \end{tabular}


\subsection{Table of trigonometric integrals}
      A list of common and not-so-common trigonometric integrals:

      \begin{tabular}{|c|c|}
        \hline
        $\int \tan x dx$ & $-\ln |\cos x|$ \\ \hline
        $\int \cot x dx$ & $\ln |\sin x|$ \\ \hline
        $\int \sec x dx$ & $\ln |\sec x + \tan x|$ \\ \hline
        $\int \csc x dx$ & $\ln |\csc x - \cot x|$ \\ \hline
        $\int \sec^2 x dx$ & $\tan x$ \\ \hline
        $\int \csc^2 x dx$ & $\cot x$ \\ \hline
        $\int \sin^n x dx$ & $\frac{-\sin^{n-1} x \cos x}{n} + \frac{n-1}{n}\int \sin^{n-2}x dx$ \\ \hline
        $\int \cos^n x dx$ & $\frac{\cos^{n-1} x \sin x}{n} + \frac{n-1}{n}\int \cos^{n-2}x dx$ \\ \hline
        $\int \arcsin x dx$ & $x \arcsin x + \sqrt{1 - x^2}$ \\ \hline
        $\int \arccos x dx$ & $x \arccos x - \sqrt{1 - x^2}$ \\ \hline
        $\int \arctan x dx$ & $x \arctan x - \frac{1}{2} \ln |1 - x^2|$ \\ \hline
      \end{tabular}


\


\subsection{Common integral substitutions} And finally, a list of common substitutions:
\resizebox{\columnwidth}{!}{%
      \begin{tabular}{|c|c|c|}
        \hline
        $\int F(\sqrt{ax + b}) dx$ & $u = \sqrt{ax + b}$ & $\frac{2}{a} \int u F(u) du$ \\ \hline
        $\int F(\sqrt{a^2 - x^2}) dx$ & $x = a \sin u$ & $a \int F(a \cos u) \cos u du$ \\ \hline
        $\int F(\sqrt{x^2 + a^2}) dx$ & $x = a \tan u$ & $a \int F(a \sec u) \sec^2 u du$ \\ \hline
        $\int F(\sqrt{x^2 - a^2}) dx$ & $x = a \sec u$ & $a \int F(a \tan u) \sec u \tan u du$ \\ \hline
        $\int F(e^{ax}) dx$ & $u = e^{ax}$ & $\frac{1}{a} \int \frac{F(u)}{u} du$ \\ \hline
        $\int F(\ln x) dx$ & $u = \ln x$ & $\int F(u) e^u du$ \\ \hline
      \end{tabular}%
      }
