\chapter{Numerical}
\kactlimport{GoldenSectionSearch.h}
\kactlimport{Polynomial.h}
\kactlimport{PolyRoots.h}
\kactlimport{PolyInterpolate.h}
\kactlimport{Lagrange.h}
\kactlimport{BerlekampMassey.h}
\kactlimport{LinearRecurrence.h}
\kactlimport{HillClimbing.h}
\kactlimport{Integrate.h}
\kactlimport{IntegrateAdaptive.h}
\kactlimport{Determinant.h}
\kactlimport{IntDeterminant.h}
\kactlimport{Elimination.h}
%\kactlimport{Simplex.h}
\kactlimport{Math-Simplex.cpp}
\kactlimport{SolveLinear.h}
\kactlimport{SolveLinear2.h}
\kactlimport{SolveLinearBinary.h}
\kactlimport{MatrixInverse.h}
\kactlimport{MatrixInverse-mod.h}
\kactlimport{Tridiagonal.h}
\kactlimport{NewtonMethod.h}
\kactlimport{NewtonSQRT.h}
%\kactlimport{MarkovChain.cpp}
\kactlimport{Polyominoes.h}
\section{Fourier transforms}
%\kactlimport{fft.cpp}
%\kactlimport{poly-exp-log.cpp}
\kactlimport{FastFourierTransform.h}
\kactlimport{FastFourierTransformMod.h}	
\kactlimport{NumberTheoreticTransform.h}
% \kactlimport{FastSubsetTransform.h}
%\kactlimport{Zeta.h}
%\kactlimport{ZetaDor.h}
\iffalse
\subsection{General linear recurrences}
  If $a_n = \sum_{k=0}^{n-1} a_k b_{n-k}$, then $A(x) = \frac{a_0}{1-B(x)}$.

\subsection{Inverse polynomial modulo $x^l$}
  Given $A(x)$, find $B(x)$ such that $A(x)B(x) = 1 + x^l Q(x)$ for some $Q(x)$.

  Step 1: Start with $B_0(x) = 1/a_0$

  Step 2: $B_{k+1}(x) = (-B_k(x)^2 A(x) + 2 B_k(x)) \mod x^{2^{k+1}}$.

\subsection{Fast subset convolution}
  Given array $a_i$ of size $2^k$ calculate $b_i = \sum_{j \BitAnd i = i} a_j$.
\fi

\


\subsection{Duality}
  max $c^Tx$ sjt to $Ax \leq b$. Dual problem is min $b^Tx$ sjt to $A^Tx \geq c$. By strong duality, min max value coincides.

\subsection{Generating functions}
  A list of generating functions for useful sequences:

  \begin{tabular}{|c|c|}
    \hline
    $(1,1,1,1,1,1,\ldots)$ & $\frac{1}{1-z}$ \\ \hline
    $(1,-1,1,-1,1,-1,\ldots)$ & $\frac{1}{1+z}$ \\ \hline
    $(1,0,1,0,1,0,\ldots)$ & $\frac{1}{1-z^2}$ \\ \hline
    $(1,0,\ldots,0,1,0,1,0,\ldots,0,1,0,\ldots)$ & $\frac{1}{1-z^2}$ \\ \hline
    $(1,2,3,4,5,6,\ldots)$ & $\frac{1}{(1-z)^2}$ \\ \hline
    $(1,\binom{m+1}{m},\binom{m+2}{m},\binom{m+3}{m},\ldots)$ & $\frac{1}{(1-z)^{m+1}}$ \\ \hline
    $(1,c,\binom{c+1}{2},\binom{c+2}{3},\ldots)$ & $\frac{1}{(1-z)^c}$ \\ \hline
    $(1,c,c^2, c^3, \ldots)$ & $\frac{1}{1-cz}$ \\ \hline
    $(0,1,\frac{1}{2},\frac{1}{3},\frac{1}{4},\ldots)$ & $\ln \frac{1}{1-z}$ \\ \hline
  \end{tabular}

\

A neat manipulation trick is:
\begin{equation*}
  \frac{1}{1-z}G(z) = \sum_{n}\sum_{k\leq n}g_kz^n
\end{equation*}

\subsection{Polyominoes} How many free (rotation, reflection), one-sided (rotation) and fixed $n$-ominoes are there?
      
      \resizebox{\columnwidth}{!}{%
       \begin{tabular}{|c|c|c|c|c|c|c|c|c|}
        \hline
        n&3&4&5&6&7&8&9&10 \\ \hline
        free&2&5&12&35&108&369&1.285&4.655 \\ \hline
        one-sided&2&7&18&60&196&704&2.500&9.189 \\ \hline
        fixed&6&19&63&216&760&2.725&9.910&36.446 \\ \hline
      \end{tabular}
      }
