\chapter{Number theory}

\section{Modular arithmetic}
	\kactlimport{ModTemplate.h}
	\kactlimport{PairNumTemplate.h}
	\kactlimport{ModInv.h}
	\kactlimport{Modpow.h}
	\kactlimport{ModSum.h}
	\kactlimport{ModMul.cpp}
	\kactlimport{ModMulLL.h}
	\kactlimport{ModSqrt.h}
	\kactlimport{MulOrder.h}
	\kactlimport{Quadratic.h}

\section{Primality}
	\kactlimport{Sieve.h}
	\kactlimport{LinearSieve.h}
	\kactlimport{MobiusSieve.h}
	\kactlimport{Mobius.h}
	\kactlimport{MillerRabin.h}
	\kactlimport{Factorize.h}
    \kactlimport{PollardRho.h}

\section{Divisibility}
	\kactlimport{ExtendedEuclidean.h}
	\kactlimport{Euclid.java}
	\kactlimport{DiophantineEquation.h}
	\kactlimport{Divisors.h}
	\kactlimport{Pell.h}
	\kactlimport{PrimeFactors.h}
	\kactlimport{NumDiv.h}
	\kactlimport{NumPF.h}
	\kactlimport{SumDiv.h}
	\kactlimport{GoldbachConjecture.cpp}
    \kactlimport{Bezout.h}
	\kactlimport{EulerPhi.h}
	\kactlimport{phiFunction.h}
	\kactlimport{DiscreteLogarithm.cpp}
	\kactlimport{Legendre.h} 
	\kactlimport{GroupOrder.h} 	

\section{Fractions}
    \kactlimport{Fractions.h}
	\kactlimport{ContinuedFractions.h}
	\kactlimport{FracBinarySearch.h}

\section{Chinese remainder theorem}
	\kactlimport{ChineseRemainder.h}
\section{Pythagorean Triples}
 The Pythagorean triples are uniquely generated by
 \[ a=k\cdot (m^{2}-n^{2}),\ \,b=k\cdot (2mn),\ \,c=k\cdot (m^{2}+n^{2}), \]
 with $m > n > 0$, $k > 0$, $m \bot n$, and either $m$ or $n$ even.

\section{Primes}
	$p=962592769$ is such that $2^{21} \mid p-1$, which may be useful. For hashing
	use 970592641 (31-bit number), 31443539979727 (45-bit), 3006703054056749
	(52-bit). There are 78498 primes less than 1\,000\,000.

	Primitive roots exist modulo any prime power $p^a$, except for $p = 2, a > 2$, and there are $\phi(\phi(p^a))$ many.
	For $p = 2, a > 2$, the group $\mathbb Z_{2^a}^\times$ is instead isomorphic to $\mathbb Z_2 \times \mathbb Z_{2^{a-2}}$.

\subsection{Primitive Roots}
	It only exists when $n$ is $2, 4, p^k, 2p^k$, where $p$ odd prime.
	If $g$ is a primitive root, all primitive roots are of the form $g^k$
	where $k,\phi(p)$ are coprime (hence there are $\phi(\phi(p))$ primitive roots).
	
\subsection{Chicken McNugget Theorem}
	Sejam $x$ e $y$ dois inteiros coprimos, o maior inteiro que não pode ser escrito como $ax + by$ é $\frac{(x-1)(y-1)}{2}$

    \subsection{Wilson's Theorem}
Seja~$n > 1$. Então~$n|(n-1)!+1$ sse~$n$ é primo. \newline
\subsection{Wolstenholme's Theorem}
Seja~$p > 3$ um número primo. Então o numerador do número 
$ 1 + \frac{1}{2}+\frac{1}{3}+\cdots+\frac{1}{p-1} $
é divisível por~$p^2$. \newline
\subsection{Bézout's identity}
For $a \neq $, $b \neq 0$, then $d=gcd(a,b)$ is the smallest positive integer for which there are integer solutions to
$$ax+by=d$$
If $(x,y)$ is one solution, then all solutions are given by
$$\left(x+\frac{kb}{\gcd(a,b)}, y-\frac{ka}{\gcd(a,b)}\right), \quad k\in\mathbb{Z}$$\newline
    \subsection{Möbius Inversion Formula}
        Se~$F(n) = \sum\limits_{d | n}{f(d)}$, então
        $f(n) = \sum\limits_{d | n}{\mu(d) F(n / d)}.$\newline
        
\subsection{Estimates}
	$\sum_{d|n} d = O(n \log \log n)$.

	The number of divisors of $n$ is at most around 100 for $n < 5e4$, 500 for $n < 1e7$, 2000 for $n < 1e10$, 200\,000 for $n < 1e19$.
	
 \subsection{Prime counting function ($\pi(x)$)} The prime counting function is asymptotic to $\frac{x}{\log x}$, by the prime number theorem.

      \ 

      \begin{tabular}{|c|c|c|c|c|c|c|c|}
        \hline
          x&10&$10^2$&$10^3$&$10^4$&$10^5$&$10^6$&$10^7$\\ \hline
          $\pi(x)$&4&25&168&1.229&9.592&78.498&664.579\\ \hline
      \end{tabular}

